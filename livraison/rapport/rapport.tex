\documentclass[11pt,a4paper]{article}
\usepackage[utf8]{inputenc}
\usepackage[french]{babel}
\usepackage{graphicx}
\usepackage{listings}
\usepackage{color}
\usepackage{hyperref}
\usepackage{geometry}
\usepackage{tikz}
\usepackage{pgf-umlsd}

\geometry{a4paper, margin=2.5cm}

\title{Rapport de Conception - Jeu de Mémoire}
\author{Votre Nom}
\date{\today}

\begin{document}

\maketitle

\tableofcontents

\section{Introduction}
Ce rapport présente la conception et l'implémentation du jeu de mémoire (MemoGame). Ce jeu permet aux joueurs de mémoriser et reproduire des formes géométriques dans un temps limité.

\section{Architecture du Projet}
Le projet est organisé selon une architecture MVC (Modèle-Vue-Contrôleur) avec les packages suivants :
\begin{itemize}
    \item \texttt{model} : Contient les classes représentant les données du jeu
    \item \texttt{view} : Contient les classes gérant l'interface utilisateur
    \item \texttt{controller} : Contient les classes gérant la logique du jeu
    \item \texttt{command} : Implémente le pattern Command pour les actions du jeu
    \item \texttt{strategy} : Implémente le pattern Strategy pour la génération de formes
    \item \texttt{evaluation} : Implémente le pattern Strategy pour l'évaluation des scores
    \item \texttt{state} : Implémente le pattern State pour gérer les différents états du jeu
\end{itemize}

\section{Design Patterns Utilisés}

\subsection{Pattern MVC (Modèle-Vue-Contrôleur)}
Le pattern MVC a été implémenté pour séparer les responsabilités dans l'application :
\begin{itemize}
    \item \textbf{Modèle} : Gère les données et la logique métier
    \item \textbf{Vue} : Affiche les données et capture les entrées utilisateur
    \item \textbf{Contrôleur} : Coordonne les interactions entre le modèle et la vue
\end{itemize}

\subsection{Pattern Command}
Le pattern Command a été utilisé pour encapsuler les actions du jeu (ajout de forme, déplacement, etc.) dans des objets. Cela permet :
\begin{itemize}
    \item L'annulation et la répétition des actions (undo/redo)
    \item La séparation entre l'action et son exécution
\end{itemize}

\subsection{Pattern Strategy}
Le pattern Strategy a été utilisé pour :
\begin{itemize}
    \item La génération de formes (RandomShapeStrategy)
    \item L'évaluation des scores (ScoreStrategy)
\end{itemize}

\subsection{Pattern State}
Le pattern State a été utilisé pour gérer les différents états du jeu :
\begin{itemize}
    \item IdleState : État initial, en attente d'interaction
    \item MovingShapeState : État lors du déplacement d'une forme
\end{itemize}

\section{Algorithmes Principaux}

\subsection{Génération de Formes}
L'algorithme de génération de formes utilise une stratégie aléatoire pour créer des formes de différentes tailles et couleurs. Les formes sont placées à des positions aléatoires sur le canvas.

\subsection{Évaluation des Scores}
L'algorithme d'évaluation des scores compare les formes dessinées par le joueur avec les formes du modèle. Il prend en compte :
\begin{itemize}
    \item La position des formes
    \item La taille des formes
    \item La couleur des formes
\end{itemize}

\subsection{Gestion du Timer}
Le timer est géré par un objet Timer de Java qui déclenche un événement après un délai spécifié. Cela permet de :
\begin{itemize}
    \item Limiter le temps de mémorisation
    \item Limiter le temps de reproduction
    \item Alterner entre les joueurs en mode deux joueurs
\end{itemize}

\section{Interface Utilisateur}
L'interface utilisateur est construite avec Swing et comprend :
\begin{itemize}
    \item Un panneau de dessin pour afficher et interagir avec les formes
    \item Des boutons pour créer différentes formes
    \item Un sélecteur de couleurs
    \item Un affichage du score et du temps restant
    \item Des boutons pour valider, annuler et répéter les actions
\end{itemize}

\section{Mode Deux Joueurs}
Le mode deux joueurs permet à deux joueurs de s'affronter en alternant les tours. Chaque joueur a un temps limité pour dessiner ou reproduire des formes.

\section{Conclusion}
Le jeu de mémoire a été conçu et implémenté en utilisant plusieurs design patterns pour assurer une architecture modulaire et extensible. L'application offre une expérience utilisateur intuitive et des fonctionnalités variées.

\appendix
\section{Captures d'écran}
% Insérer ici les captures d'écran de l'application

\end{document} 